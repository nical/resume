\documentclass[a4paper, 18pt]{article}

% Faire des marges un peu moins large que celles par défaut
\usepackage[top=20mm, bottom=20mm, left=18mm, right=18mm]{geometry}
\usepackage{ucs}
\usepackage[utf8x]{inputenc} % Pour l'encodage 
% Reconnaitre les caratères accentués dans le source.
\usepackage[T1]{fontenc} 
% Meilleurs polices
%\usepackage{concmath}
\usepackage{lmodern}
\usepackage[francais]{babel}
% Insertion d'images
\usepackage{graphicx}
% Pour le listing de code
%\usepackage{listings}
% Pour la coloration syntaxique
\usepackage{xcolor}
% Pour fixer l'interlignage
\usepackage{setspace} 
% Pour faire un index (ici glossaire)
%\usepackage{makeidx}
% Pour gérer les liens internes et les URL cliquables
\usepackage{url}
% Pour les headers et footers
%\usepackage{fancyhdr}
% Pour le logo en haut a droite
%\usepackage{eso-pic} 
% Pour l'enroulement du texte autour des figures
%\usepackage{wrapfig}
% Pour la couverture en PDF pleine page
%\usepackage{pdfpages} 
% Pour la biblio bibtex
\usepackage{bibunits}
% Pour gérer les éléments flottants
\usepackage{float}
% Pour les cadres à ombrage du glossaire
\usepackage{fancybox}
% Pour faire des sous-figures correctement numérotés
\usepackage{subfigure}
% Pour mettre les liens cliquables
\usepackage{hyperref}
\NoAutoSpaceBeforeFDP



\setlength{\headheight}{14.5pt}

% Couleur des url et des liens internes.
\hypersetup{urlcolor=blue,linkcolor=black,citecolor=black,colorlinks=true}


